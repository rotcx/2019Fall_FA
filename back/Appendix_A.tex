% !TeX root = ../2019Fall_FA.tex
% This is the Appendix A.

\appendix
	
	\chapter{一些因为太显然而没有在课上证明的命题}
	
	\section{拓扑空间与度量空间}
	
	\textbf{命题\,\,\ref{prop:Hausdorff空间的相关命题1}}\ \ Hausdorff空间的一部分性质:
	\begin{enumerate}[(1)]
		\item  $ E $ 是 Hausdorff空间 $ \Longleftrightarrow $ $ \forall x\in E $, 其所有闭邻域的交为 $\{ x \}$
	    \item $ E $ 是 Hausdorff空间 $ \Longrightarrow $ $\bigcap \CN(x)=\{x\}$;
	    \item $ E $ 是 Hausdorff空间 $ \Longrightarrow $ $ \forall F\subset E $, $ F $ 也是Hausdorff空间. 
	\end{enumerate}
	\begin{Proof}
	(1) \textsl{必要性}. 用$ \mathcal{E}(x) $记$ x $的所有闭邻域, 则$ x\in\bigcap\mathcal{E}(x) $是显然的. 若$ \exists x'\in\bigcap\mathcal{E}(x) $且$ x'\ne x $, 由Hausdorff空间的定义可知
	\[
	\exists U\in\CN({x})\,\exists V\in\CN({x'})\,(U\cap V=\varnothing),
	\]
	并且可以要求$ U,\ V $都是开集, 则$ V^c\in\mathcal{E}(x) $但$ x'\notin V^c $, 矛盾.
	
	\textsl{充分性}. 显然
	
	(2) 其中$ \{x\}\subset\bigcap\CN({x}) $是显然的, 由(1)知
	\[
	\bigcap\CN(x)\subset\bigcap\mathcal{E}(x)=\{x\},
	\]
	从而只能$ \{x\}=\bigcap\CN({x}) $.
	
	(3) 显然.	\qed
	\end{Proof}
	
	\textbf{注1.2.2的\,\,\ref{item:Hausdorff空间上极限唯一性}}\ \ Hausdorff空间中的序列至多只有一个极限.
	
	我们希望证明一个更强的命题, 尽管实分析课程中已经介绍过相关概念, (但是我猜可能考完试就忘记了), 所以在这里也再次重复一遍:
	\begin{Definition}
		对一般的准序集$ (P, \lesssim) $, 若$ \forall x_1, x_2\in P, \exists x\in P\,((x_1\lesssim x)\land (x_2\lesssim x)) $, 则称$ P $是\textbf{上定向}的; 若$ \forall x_1, x_2\in P, \exists x\in P\,((x_1\gtrsim x)\land (x_2\gtrsim x)) $, 则称$ P $是\textbf{下定向}的. 若$ P $上定向或者下定向, 则称$ P $是一个\textbf{定向集}.
		
		当$ \alpha $是一个定向集时, 一个$ \alpha $-组称为一个\textbf{网}. 当$ (x_i)_{i\uparrow\alpha} $是$ (P,\lesssim) $上的一个网时, 形式地定义\textbf{上极限}
		\[
		\limsup_{i\uparrow\alpha}x_i=\varlimsup_{i\uparrow\alpha}x_i:=\inf_{j\in\alpha}\sup_{i\gtrsim j}x_i
		\]
		和\textbf{下极限}
		\[
		\liminf_{i\uparrow\alpha}x_i=\varliminf_{i\uparrow\alpha}x_i:=\sup_{j\in\alpha}\inf_{i\gtrsim j}x_i.
		\]
		特别地, 当$ \limsup\limits_{i\uparrow\alpha}x_i=\liminf\limits_{i\uparrow\alpha}x_i $时, 称$ \lim\limits_{i\uparrow\alpha}:=\limsup\limits_{i\uparrow\alpha} $为$ (x_i)_{i\uparrow\alpha} $的\textbf{极限}.
	\end{Definition}
	\begin{Proof}
	下面我们证明\textsl{Hausdorff空间上的网至多只有一个极限}. 设$ \alpha $是一个上定向集, $ (x_i)_{i\uparrow\alpha} $是Hausdorff空间$ E $上的网, 并设$ x\ne y $使得
	\[
	\left(\lim_{i\uparrow\alpha}x_i=x\right)\land\left(\lim_{i\uparrow\alpha}x_i=y\right),
	\]
	则由Hausdorff空间的定义, 有
	\[
	\exists U\in\CN(x)\,\exists V\in\CN({y})\,(U\cap V=\varnothing),
	\]
	而
	\[
	\exists i_1\in\alpha( \forall j\gtrsim i_1 : x_j\in U )\land\exists i_2\in\alpha( \forall j\gtrsim i_2 : x_j\in V ),
	\]
	矛盾.\qed
	\end{Proof}
	
	\textbf{命题\,\,\ref{prop:Cauchy列的性质}}\ \ Cauchy列有以下性质:
		\begin{enumerate}[(1)]
	    	\item 若 $ (x_{n})_{n\geqslant1} $ 是收敛列, 则 $ (x_{n})_{n\geqslant1} $ 是Cauchy列;
	        \item 若 $ (x_{n})_{n\geqslant1} $ 是Cauchy列且有收敛子列, 则 $ (x_{n})_{n\geqslant1} $ 是收敛列;
	        \item 若 $ (x_{n})_{n\geqslant1} $ 是Cauchy列, 则 $ (x_{n})_{n\geqslant1} $ 有界. (即 $ \exists x\in E\,\exists r>0\,((x_{n})_{n\geqslant1} \subset B(x, r)) $ )
	   \end{enumerate}
	\begin{Proof}
	(1) 设$ \lim\limits_{n\to\infty}x_n=x_0 $, 则
	\[
	\forall\varepsilon>0\,\exists n_0\in\N\,\left(n\geqslant n_0\Rightarrow d(x_n,x_0)<\frac{\varepsilon}{2}\right)
	\]
	则对上述$ \varepsilon>0 $, 当$ n,m\geqslant n_0 $时有
	\[
	d(x_n,x_m)\leqslant d(x_n,x_0)+d(x_0,x_m)<\frac{\varepsilon}{2}+\frac{\varepsilon}{2}=\varepsilon,
	\]
	即$ (x_n)_{n\geqslant 1} $是Cauchy列.
	
	(2) 设$ (x_{n_k})_{k\geqslant 1}\subset(x_n)_{n\geqslant 1} $收敛到$ x_0 $, 即
	\[
	\forall\varepsilon>0\,\exists k_0\in\N\,\left(k\geqslant k_0\Rightarrow d(x_{n_k},x_0)<\frac{\varepsilon}{2}\right)
	\]
	而由$ (x_n)_{n\geqslant 1} $是Cauchy列可知
	\[
	\forall\varepsilon>0\,\exists n_0\in\N\,\left( n\geqslant n_0\Rightarrow\forall p\in\N,\,d(x_{n+p},x_n)<\frac{\varepsilon}{2} \right)
	\]
	因此$ n>n_k $时取$ p=n_l-n\in\N $, 则
	\[
	d(x_n,x_0)\leqslant d(x_n,x_{n+p})+d(x_{n+p},x_0)<\frac{\varepsilon}{2}+\frac{\varepsilon}{2}=\varepsilon.
	\]
	即$ \lim\limits_{n\to\infty}x_n=x_0 $.
	
	(3) 设$ (x_n)_{n\geqslant 1} $是Cauchy列, 则取$ \varepsilon=1 $, 有
	\[
	\exists k\in\N\,(n,m\geqslant k\Rightarrow d(x_n,x_m)<1),
	\]
	则$ \forall n\geqslant k $, 有$ d(x_n,x_k)<1 $. 从而取$ r=\left(\max\limits_{1\leqslant i\leqslant k}d(x_i,x_k)\right)\land 1 $即可.\qed
	\end{Proof}
	
	\textbf{命题\,\,\ref{prop:连续映射的性质}}\ \ 连续映射具有以下性质:
		\begin{enumerate}[(1)]
	    \item $ f $ 在 $ E $ 上连续 $ \Longleftrightarrow $ 对 $ F $ 上的开集 $ V $ , 有 $ f^{-1}(V) $ 是 $ E $ 中开集;
	    \item $ f $ 在 $ E $ 上连续 $ \Longleftrightarrow $ 对 $ F $ 上的闭集 $ B $ , 有 $ f^{-1}(B) $ 是 $ E $ 中闭集;
	    \item 连续映射的复合是连续映射;
	    \item $ \tau $, $ \tau' $ 是 $ E $ 中拓扑,  $ \tau $ 是 $ \tau' $ 的强拓扑 $ \Longleftrightarrow $ $ \mathrm{id}_{E}:(E, \tau)\to(E, \tau') $ 连续. 
		\end{enumerate}
	\begin{Proof}
	(1) \textsl{必要性.}\ \ 由$ f $在$ E $上连续可知
	\[
	\forall x\in E\,\forall V\in\CN(f(x))\,(f^{-1}(V)\in\CN(x)),
	\]
	从而任取$ V $是$ F $中开集, $ \forall x\in f^{-1}(V) $, 都存在$ x $的开邻域$ U $使得$ U\subset f^{-1}(V) $, 从而$ f^{-1}(V) $也是开集.
	
	\textsl{充分性.} 若$ f^{-1}(V) $也是开集, 注意到$ \forall x\in E $使得$ V\in\CN(f(x)) $有$ x\in f^{-1}(V) $. 从而$ f $在$ E $连续.
	
	(2) 由$ f^{-1}(B^c)=(f^{-1}(B))^c $可知成立.
	
	(3) 由(1)的结论可知.
	
	(4) 因为$ \tau $是$ \tau' $的强拓扑, 故$ \tau' $-开集一定是$ \tau $-开集. 故任取$ \tau' $-开集$ V $, 总有$ \id_E^{-1}(V)=V $是$ \tau $-开集, 故$ \id_E $连续.\qed
	\end{Proof}
	
	\textbf{注\,\,\ref{rmk:预紧性的刻画}}\ \ 有关预紧性的刻画, 由定理\ref{thm:紧等价}中的``(3) $ \Leftrightarrow $ (4)"可知
	\begin{enumerate}[(1)]
	\item $ E $是预紧的$ \Longleftrightarrow $ $ E $的任一序列存在Cauchy子列;
	\item $ A $相对紧$ \Longleftrightarrow $ $ A $中无穷序列存在子列收敛到$ E $中的元素;
	\item 若$ E $是完备的, 则$ A $相对紧$ \Longleftrightarrow $ $ A $预紧.
	\end{enumerate}
	\begin{Proof}
	(2) \textsl{必要性.} 由$ A $相对紧至$ \bar{A} $紧, 从而$ \bar{A} $序列紧, 即
	\[
	\forall (x_n)_{n\geqslant 1}\subset\bar{A}\,\exists x\in\bar{A}\subset E\,(\lim_{k\to\infty}x_{n_k}=x),
	\]
	取$ (\tilde{x}_{n_k})_{k\geqslant 1} $是$ (x_{n_k})_{k\geqslant 1} $中去掉所有$ \partial A $中元素构成的序列即可.
	
	\textsl{充分性.} 若$ \forall (x_n)_{n\geqslant 1}\subset A $, 都有$ (x_{n_k})_{n\geqslant 1}\subset(x_n)_{n\geqslant 1} $使得存在$ x\in E $满足$ \lim\limits_{k\to\infty}x_{n_k}=x $, 从而$ x\in\bar{A} $. 从而$ \bar{A} $是序列紧的, 故$ \bar{A} $紧, 也即$ A $相对紧.
	
	(3) 由$ E $完备可知
	\[
	A\,\text{相对紧}\,\Longleftrightarrow\bar{A}\,\text{紧}\,\Longleftrightarrow\bar{A}\,\text{预紧}\,\Longleftrightarrow A\,\text{预紧}.
	\]
	\qed
	\end{Proof}
	
	\textbf{命题\,\,\ref{prop:乘积拓扑空间的继承性质}}\ \ 设$ (E_i)_{i\in\alpha} $是一族拓扑空间 $ E=\prod\limits_{i\in\alpha}E_i $是乘积拓扑空间.
		\begin{enumerate}[(1)]
		\item 若$ \forall i\in\alpha $, $ E_i $是Hausdorff空间, 则$ E $也是Hausdorff空间.
		
		\item 若$ \forall i\in\alpha $, $ E_i $是紧空间, 则$ E $也是紧空间.
		
		\item 设$ \alpha=\N^\ast $, 若$ \forall i\in\alpha,\ E_i $是可度量化的, 则$ E $是可度量化的. (可度量化是指存在$ E $上的度量$ d $使得$ d $诱导的拓扑与$ E $上原本定义的拓扑一致)
		\end{enumerate}
	\begin{Proof}
	(1) 设$ x=(x_i)_{i\in\alpha} $与$ y=(y_i)_{i\in\alpha} $是$ E $中两个不同的元素, 那么存在$ i_0\in\alpha $使得$ x_{i_0}\ne y_{i_0} $. 由$ E_{i_0} $是Hausdorff空间可知
	\[
	\exists U_{i_0}\in\CN(x_{i_0})\,\exists V_{i_0}\in\CN(y_{i_0})\,(U_{i_0}\cap V_{i_0}=\varnothing).
	\]
	于是可以取$ E $中两个开集:
	\[
	U=U_{i_0}\times\prod_{i\in\alpha\sm\{i_0\}}E_i,\qquad V=V_{i_0}\times\prod_{i\in\alpha\sm\{i_0\}}E_i
	\]
	使得$ U\in\CN(x) $, $ V\in\CN(y) $但是$ U\cap V=\varnothing $. 从而$ E $也是Hausdorff空间.
	
	(2) 这一结论称为Tychonoff定理, 它的证明不仅超出了本课程的要求范围, 也超出了一般的\textsl{Topology I}课程的要求范围. 在$ \alpha $是无限集的情形需要借助选择公理进行证明, 因此在此不再证明这一定理, 而只证明$ \alpha $是有限集的情形. 事实上即使是有限情形的证明也不是trivial的, 它本应放在正文中, 但是为了这一命题的3个结论都在这里, 将其调整到这里来证明. 由归纳法只需要证明$ \alpha=\{1,2\} $的情形是成立的即可.
	
	设 $ (O_{i})_{i\in\alpha} $ 是 $ E $ 上的开覆盖, 则
	\[
		\forall x=(x_{1}, x_{2})\in E\,\exists i_{x}\in\alpha\,(x\in O_{i_{x}}).
	\]
	由乘积拓扑的定义可以知道存在 $ V_{x} $, $ W_{x} $ 分别是 $ x_{1} $和 $ x_{2} $ 关于对应拓扑上的开邻域, 使得 $ V_{x}\times W_{x}\subset O_{i_{x}} $. 固定 $ x_{2}\in E_{2} $, 设 $ F=E_{1}\times \{ x_{2} \} $, 则 $ (V_{x})_{x\in F} $ 是 $ E_{1} $ 上的一个开覆盖. 由于 $ E_{1} $ 紧, 则存在有限集 $ J_{x_{2}}\in \fin F $, 使得 $ \bigcup_{x\in J_{x_{2}}}V_{x}=E_{1} $. 接下来我们令
	\[
		A_{x_{2}}=\bigcap_{x\in J_{x_{2}}}W_{x},
	\]
	则 $ A_{x_{2}} $ 是含有 $ x_{2} $ 的开集. 当 $ x_{2} $ 在 $ E_{2} $ 中变动时, 可以发现 $ (A_{x_{2}})_{x_{2}\in E_{2}} $ 也是 $ E_{2} $ 的开覆盖, 同样由于 $ E_{2} $ 紧, 则存在一个有限集 $ K\in\fin E_{2} $, 使得 $ \bigcup_{x_{2}\in K}A_{x_{2}}=E_{2} $. 由此可得
	\[
		E=\bigcup_{x\in J} V_{x}\times W_{x},\qquad J=\bigcup_{x_{2}\in K}J_{x_{2}}.
	\]
	事实上由以上讨论可以知道, 当我们任取一个 $ a=(a_{1}, a_{2})\in E $, 必存在某个 $ x_{2}\in K $, 使得 $ a_{2}\in A_{x_{2}} $, 那么存在 $ x\in J_{x_{2}} $, 使得 $ a_{1}\in V_{x} $; 这里确定的 $ x\in J $, 且 $ a\in V_{x}\times W_{x} $, 由此可见,  $ E $ 被 $ (O_{i_{x}})_{i_{x}\in J} $ 覆盖, 因为 $ J $ 是有限集, 则 $ E $ 是紧的.
	
	(3) 因为$ d $与$ \min\{d,1\} $诱导相同的拓扑, 不妨假设每个$ E_n $上的度量$ d_n\leqslant 1 $. 那么对$ E $中的两个元素$ x=(x_n)_{n\geqslant 1} $, $ y=(y_n)_{n\geqslant 1} $, 可取
	\[
	d(x,y)=\sup_{n\geqslant 1}\frac{d_n(x_n,y_n)}{n}.
	\]
	容易证明$ d $的确是$ E $上的度量. 用$ B $记依度量$ d $的开球而用$ B_n $记依度量$ d_n $的开球, 那么由$ d $的定义可知
	\[
	B\left(x,\frac{1}{k}\right)=\prod_{n=1}^kB_n\left(x_n,\frac{n}{k}\right)\times\prod_{n>k}E_n
	\]
	是基础开集, 因此$ (B(x,1/k))_{k\geqslant 1} $是$ x $的邻域基, 因此$ E $上的乘积拓扑可由$ d $诱导, 也即$ E $可度量化.\qed
	     
	\end{Proof}
	
	\textbf{推论\,\,\ref{cor:距离函数连续}}\ \ 设$ (E,d) $是度量空间, $ \forall A\subset E $, 映射$ x\mapsto d(x,A) $是连续的.
	\begin{Proof}
	由$ d(x,A) $的定义可知对$ x,y\in E $, 有
	\[
	\forall y\in E\,\exists z\in A\,(d(y,z)<d(y,A)+\varepsilon),
	\]
	从而
	\[
	d(x,A)\leqslant d(x,z)\leqslant d(x,y)+d(y,z)<d(x,y)+d(y,A)+\varepsilon,
	\]
	即
	\[
	d(x,A)-d(y,A)<d(x,y)+\varepsilon.
	\]
	令$ \varepsilon\to 0^+ $就有$ d(x,A)-d(y,A)\leqslant d(x,y) $. 由对称性可证$ d(y,A)-d(x,A)\leqslant d(x,y) $, 从而
	\[
	\abs{d(x,A)-d(y,A)}\leqslant d(x,y),
	\]
	也即$ x\mapsto d(x,A) $是Lipschitz的, 从而它连续.\qed
	\end{Proof}
	
	\textbf{定理\,\ref{thm:Ascoli}\,证明补充}\ \ 在本定理必要性证明中使用了这样一个结论: 若$ \CH\subset C(K,E) $相对紧, 那么$ \baro{\CH}(x)=\baro{\CH(x)} $.
	\begin{Proof}
	由$ \CH\subset C(K,E) $相对紧可知$ \baro{\CH} $紧, 由映射
	\[
	\rho_x : C(K,E)\to E,\qquad f\mapsto f(x)
	\]
	连续可知$ \baro{\CH}(x) $紧. 又因为$ E $是Hausdorff的, 从而$ \baro{\CH}(x) $闭. 从而由$ \CH(x)\subset\baro{\CH}(x) $可知$ \baro{\CH(x)}\subset\baro\CH(x) $. 而$ \forall f\in\baro\CH $, 存在$ (f_n)_{n\geqslant1}\subset\CH $使得$ f_n\to f $, 由$ \rho_x $的连续性可知$ f(x)\in\baro{\CH(x)} $, 于是$ \baro\CH(x)\subset\baro{\CH(x)} $, 故$ \baro{\CH}(x)=\baro{\CH(x)} $.\qed
	\end{Proof}
	
	\section{线性算子与线性泛函}
	
	\textbf{定义\,\ref{def:Hilbert空间}\,下的注记}\ \ 内积空间的完备化是Hilbert空间.
	\begin{Proof}
	设内积空间$ H $上的内积$ \lrangle{\cdot,\cdot} $诱导范数$ \norm{\cdot} $, 范数$ \norm{\cdot} $诱导度量$ d(\cdot,\cdot) $, 对度量空间$ (H,d) $作完备化$ (\hat{H},d) $, 那么任取$ H $中一Cauchy列$ (x_n)_{n\geqslant 1} $, 存在$ \hat{x}\in\hat{H} $使得$ d(x_n,\hat{x})\to 0 $成立.
	
	定义$ \hat{H} $上的范数$ \norm{x-y}:=d(x,y) $, 这一定义在$ H $上的限制即为$ H $中的范数, 由度量的连续性可知$ \norm{x_n}\to\norm{x} $. 再由范数可以定义一个内积
	\[
	\lrangle{x,y}:=\begin{cases}
	\frac{1}{4}\sum_{k=0}^3\imag^k\norm{x+\imag^ky}^2 & ,\K=\C\\
	\frac{1}{2}(\norm{x+y}^2-\norm{x}^2-\norm{y}^2) & ,\K=\R
	\end{cases}
	\]
	这一定义在$ H $上的限制即为$ H $中的内积, 那么由范数的连续性可知$ \lrangle{x_n,y}\to\lrangle{x,y} $, 即$ \hat{H} $上是一个完备的内积空间, 即是一个Hilbert空间.\qed
	\end{Proof}
	
	\textbf{注\,\ref{rmk:正交性与正交补}\,的(2)}\ \ $ A^\bot=\bar{A}^\bot=(\Span A)^\bot=(\baro{\Span A})^\bot $.
	\begin{Proof}
	由定义可知
	\[
	A^\bot=\{ y : \lrangle{x,y}=0,\ \forall x\in A \},
	\]
	从而由内积的连续性可知$ A^\bot=\bar{A}^\bot $, 由内积的线性性可知$ A^\bot=(\Span A)^\bot $. 而$ \bar{A}^\bot=(\Span \bar{A})^\bot=(\baro{\Span A})^\bot $, 结论得证.\qed
	\end{Proof}
	