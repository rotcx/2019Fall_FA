% !TeX root = main.tex
% This is the Appendix C.

\chapter{《泛函分析基础》期末试题}

	\textbf{2019年秋季学期\quad 泛函分析基础\quad 期末试题}

	本试卷共\textbf{9}道大题, \textbf{1}道附加题. 总分100分, 考生的最终成绩取9道大题和附加题分数之和与100的最小值. 考试时间\textbf{120}分钟. 除定理公式所涉及的人名以外, 请使用\textbf{中文}. 大题之间是相互独立的, 后面的问题可以使用前面的结论, 无论答题者是否已经得到正确的证明或答案.

\begin{enumerate}
	\item (10分)设$ f, g\in L_2(0,1) $, 证明:
	\[
	\norm{f+g}_2^2+\norm{f-g}_2^2=2(\norm{f}_2^2+\norm{g}_2^2).
	\]
	\item (10分)设$ H $是Hilbert空间, $ u\in\CB(H) $. 证明: $ \norm{u}=\norm{\Star u} $.
	\item (15分)证明无理数集$ \J $是$ \Gd $集, 且不是$ \Fs $集, 并举例说明一个集合可以既是$ \Gd $集也是$ \Fs $集.
	\item (10分)证明: Hilbert空间是自反的.
	\item (5分)设$ E, F $都是Banach空间, 证明紧算子全体$ \CK(E,F) $构成向量空间.
	\item (10分)设$ A\subset\ell_4 $, 且$ x=(x_n)_{n\geqslant 1} $满足$ \abs{x_n}\leqslant1/\sqrt{n} $, 证明$ A $是$ \ell_4 $中的预紧集.
	\item (10分)设$ 1<p\leqslant \infty $, 考虑$ \R^3 $上的$ p $范数:
	\[
	\norm{(x_1,x_2,x_3)}_p=\begin{cases}
	(\abs{x_1}^p+\abs{x_2}^p+\abs{x_3}^p)^{1/p} & ,p<\infty\\
	\max\{ \abs{x_1},\abs{x_2},\abs{x_3} \} &,p=\infty
	\end{cases}
	\]
	设$ F=\R\times\{0\}\times\{0\} $, 即由$ e_1=(1,0,0) $生成的向量子空间, 并设$ f : F\to\R $是线性泛函, 满足$ f(e_1)=1 $. 请确定所有从$ F $到$ \R^3 $的保范延拓.
	\item (20分) 考虑$ E=(C[0,1],\norm{\cdot}_\infty) $, 且$ E $中均为Lipschitz函数.

	\hspace{4em}(1) 设 $ x, y\in[0, 1] $ 且 $ x\ne y $, 定义泛函 $ \varPhi_{x, y}:E\to\R $ 为
		\[
			\varPhi_{x, y}(f)=\frac{f(y)-f(x)}{y-x}.
		\]
	证明 $\{ \varPhi_{x, y}:x, y\in[0, 1], x\ne y \}$ 是 $ \Star{E} $ 中的有界集.

	\hspace{4em}(2) 导出 $ E $ 中的闭单位球在 $ [0, 1] $ 上等度连续, 且 $ \dim E<\infty $.
	
	\item (10分) 设$ E $是Banach空间, $ B\subset\Star{E} $, 证明: $ B $是相对$ \Star{w} $--紧的当且仅当$ B $是有界的.
	\item (5分, 附加题) 设$ E, F $都是Banach空间, $ u\in\CB(E,F) $并满足$ u(B_E) $在$ B_F $中稠密, 证明: $ u(B_E)=B_F $.
\end{enumerate}
\newpage
	\textbf{2020年秋季学期\quad 泛函分析基础\quad 期末试题}

	本试卷共\textbf{10}道大题, \textbf{1}道附加题. 总分100分, 考生的最终成绩取10道大题和附加题分数之和与100的最小值. 考试时间\textbf{120}分钟. 除定理公式所涉及的人名以外, 请使用\textbf{中文}. 大题之间是相互独立的, 后面的问题可以使用前面的结论, 无论答题者是否已经得到正确的证明或答案.

\begin{enumerate}
	\item (12分) 设 $ f : \R^2\to \R $ 一致连续, 证明: 存在非负常数 $ a, b $使得 $ \abs{f(x)}\leqslant a\norm{x}_4+b $, 其中 $ \norm{x}_4=(x_1^4+x_2^4)^{1/4} $.
	\item (12分) 证明: 赋范空间是 Banach 空间的充分必要条件是其上的绝对收敛级数均收敛.
	\item (12分) 设 $ B\subset E $ 是赋范空间 $ E $ 中的平衡闭凸集, $ x_0\notin B $. 证明: 存在 $ f\in\Star{E} $ 使得 $ f(x_0)=1 $ 且 $ \sup_{x\in B}\abs{f(x)}<1/2 $.
	\item (8分) 证明: $ \ell_2 $ 是自反空间.
	\item (10分) 证明: 有限秩算子空间构成一个向量空间.
	\item (11分) 设 $ (\varOmega,\CA,\mu) $ 是有限测度空间. 证明: 若 $ 0<p<q\leqslant\infty $, 则 $ L_q(\varOmega)\subset L_p(\varOmega) $. 并举例说明题设条件中``有限测度空间''中的``有限''条件是必要的.
	\item (10分) 证明: $ \ell_5 $ 上没有等价的内积范数.
	\item (8分) 举例说明可分赋范空间的对偶空间未必可分, 并证明举出的例子.
	\item (10分) 设 $ E $ 是实向量空间,
	
	\hspace{4em}(1) 当 $ \varOmega $ 是 $ E $ 中的吸收凸集时, 证明: Minkowski泛函 $ p_\varOmega $ 是次线性泛函;

	\hspace{4em}(2) 在 (1) 的前提下, 当 $ E $ 是实赋范空间而 $ \varOmega $ 是开集时, 证明: $ \varOmega=\{x\in E : p_\varOmega(x)<1\} $.

	\item (7分) 设 $ e_n=\exp(\imag nt), t\in[0,2\pi] $, $ f_n=(1/n)\sum_{k=1}^{3n^2}e_k $. 证明: $ (f_n)_{n\geqslant 1} $ 在 $ L_2[0,2\pi] $ 中弱收敛到0, 但不依范数收敛到0.
	\item (5分, 附加题) 证明: 局部紧的 Hausdorff 空间是 Baire 空间.
\end{enumerate}